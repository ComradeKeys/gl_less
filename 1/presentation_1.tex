%%%%%%%%%%%%%%%%%%%%%%%%%%%%%%%%%%%%%%%%%
% Beamer Presentation
% LaTeX Template
% Version 1.0 (10/11/12)
%
% This template has been downloaded from:
% http://www.LaTeXTemplates.com
%
% License:
% CC BY-NC-SA 3.0 (http://creativecommons.org/licenses/by-nc-sa/3.0/)
%
%%%%%%%%%%%%%%%%%%%%%%%%%%%%%%%%%%%%%%%%%

%----------------------------------------------------------------------------------------
%	PACKAGES AND THEMES
%----------------------------------------------------------------------------------------

\documentclass{beamer}

\mode<presentation> {

% The Beamer class comes with a number of default slide themes
% which change the colors and layouts of slides. Below this is a list
% of all the themes, uncomment each in turn to see what they look like.

%\usetheme{default}
%\usetheme{AnnArbor}
%\usetheme{Antibes}
%\usetheme{Bergen}
\usetheme{Berkeley}
%\usetheme{Berlin}
%\usetheme{Boadilla}
%\usetheme{CambridgeUS}
%\usetheme{Copenhagen}
%\usetheme{Darmstadt}
%\usetheme{Dresden}
%\usetheme{Frankfurt}
%\usetheme{Goettingen}
%\usetheme{Hannover}
%\usetheme{Ilmenau}
%\usetheme{JuanLesPins}
%\usetheme{Luebeck}
%\usetheme{Madrid}
%\usetheme{Malmoe}
%\usetheme{Marburg}
%\usetheme{Montpellier}
%\usetheme{PaloAlto}
%\usetheme{Pittsburgh}
%\usetheme{Rochester}
%\usetheme{Singapore}
%\usetheme{Szeged}
%\usetheme{Warsaw}

% As well as themes, the Beamer class has a number of color themes
% for any slide theme. Uncomment each of these in turn to see how it
% changes the colors of your current slide theme.

%\usecolortheme{albatross}
%\usecolortheme{beaver}
%\usecolortheme{beetle}
%\usecolortheme{crane}
%\usecolortheme{dolphin}
%\usecolortheme{dove}
%\usecolortheme{fly}
%\usecolortheme{lily}
%\usecolortheme{orchid}
%\usecolortheme{rose}
%\usecolortheme{seagull}
%\usecolortheme{seahorse}
%\usecolortheme{whale}
%\usecolortheme{wolverine}

%\setbeamertemplate{footline} % To remove the footer line in all slides uncomment this line
%\setbeamertemplate{footline}[page number] % To replace the footer line in all slides with a simple slide count uncomment this line

%\setbeamertemplate{navigation symbols}{} % To remove the navigation symbols from the bottom of all slides uncomment this line
}

\usepackage{graphicx} % Allows including images
\usepackage{booktabs} % Allows the use of \toprule, \midrule and \bottomrule in tables

%----------------------------------------------------------------------------------------
%	TITLE PAGE
%----------------------------------------------------------------------------------------

\title[Hello Triangle]{Intro to Opengl} % The short title appears at the bottom of every slide, the full title is only on the title page

\author{Brigham H. Keys, Esq., Dane Christensen} % Your name
\institute[BYU-I] % Your institution as it will appear on the bottom of every slide, may be shorthand to save space
{
BYU-Idaho \\ % Your institution for the title page
\medskip
\textit{key13005@byui.edu} % Your email address
}
\date{\today} % Date, can be changed to a custom date

\begin{document}

\begin{frame}
\titlepage % Print the title page as the first slide
\end{frame}

%\begin{frame}
%\frametitle{Table of Contents} % Table of contents slide, comment this block out to remove it
%\tableofcontents % Throughout your presentation, if you choose to use \section{} and \subsection{} commands, these will automatically be printed on this slide as an overview of your presentation
%\end{frame}

%----------------------------------------------------------------------------------------
%	PRESENTATION SLIDES
%----------------------------------------------------------------------------------------

%------------------------------------------------
\section{What is OpenGL?} % Sections can be created in order to organize your presentation into discrete blocks, all sections and subsections are automatically printed in the table of contents as an overview of the talk
%------------------------------------------------

\subsection{What am I going to need?} % A subsection can be created just before a set of slides with a common theme to further break down your presentation into chunks

\begin{frame}
\frametitle{What is OpenGL?}
OpenGL is the industry's foundation for high performance graphics, from games to virtual reality, mobile phones, and supercomputers. It is a cross-language, cross-platform application programming interface for rendering 2D and 3D vector graphics. The API is typically used to interact with a graphics processing unit (GPU), to achieve hardware-accelerated rendering. It does not handle windows, keyboard, mouse or any other kind of input from the user. However there are other APIs that work next to OpenGL to accomplish these tasks. %Needs a citation to wikipedia
\end{frame}

%------------------------------------------------

%\begin{frame}
%\frametitle{History of OpenGl}
%\begin{itemize}
%\item
%\end{itemize}
%\end{frame}
\subsection{Libraries used}
%------------------------------------------------

\begin{frame}
\frametitle{Libraries used in this course}
\begin{block}{OpenGL}
Chances are you already have the OpenGL installed as it is often bundled with your graphics card driver. However you also do not likely have the development files we are going to use installed.
\end{block}

\begin{block}{The OpenGL Utility Library (GLU)}
Is a collection of functions that do drawing on a higher level, and has functions that generally make our lives easier. It is bundled with OpenGL and uses OpenGL directly.
\end{block}

\begin{block}{FreeGlut}
Another collection of functions that have high level drawing. But it's main purpose for us is to handle the keyboard, mouse and window handling. This will likely need to be installed.
\end{block}
\end{frame}

%------------------------------------------------
\section{Hello world Triangle}
\subsection{Setting up the compiler}
%------------------------------------------------

\begin{frame}
\frametitle{Setting up the compiler}
To compile code containing OpenGL functions, we need to let the compiler know that we are calling them in the first place by linking the libraries to our binary. Below are the individual flags we need to give to the compiler to use these utilities. Here is the syntax to compile the code from the command line:
\begin{table}
\begin{tabular}{l l l l l}
  \toprule
  \textbf{Compiler} & \textbf{File} & \textbf{OpenGL} & \textbf{GLU} & \textbf{FreeGlut}\\
  \midrule
  g++ & main.cpp & -lGL & -lGLU & -lGLUT \\
  \bottomrule
\end{tabular}
\end{table}
\end{frame}

%------------------------------------------------

\begin{frame}
  \frametitle{Initialize OpenGL}
\end{frame}

%------------------------------------------------

\begin{frame}[fragile] % Need to use the fragile option when verbatim is used in the slide
  \frametitle{Verbatim}
\begin{example}[Theorem Slide Code]
\begin{verbatim}
\begin{frame}
\frametitle{Theorem}
\begin{theorem}[Mass--energy equivalence]
$E = mc^2$
\end{theorem}
\end{frame}\end{verbatim}
\end{example}
\end{frame}

%------------------------------------------------

\begin{frame}
\frametitle{Figure}
Uncomment the code on this slide to include your own image from the same directory as the template .TeX file.
%\begin{figure}
%\includegraphics[width=0.8\linewidth]{test}
%\end{figure}
\end{frame}

%------------------------------------------------

\begin{frame}[fragile] % Need to use the fragile option when verbatim is used in the slide
\frametitle{Citation}
An example of the \verb|\cite| command to cite within the presentation:\\~

This statement requires citation \cite{p1}.
\end{frame}

%------------------------------------------------

\begin{frame}
\frametitle{References}
\footnotesize{
\begin{thebibliography}{99} % Beamer does not support BibTeX so references must be inserted manually as below
\bibitem[Smith, 2012]{p1} John Smith (2012)
\newblock Title of the publication
\newblock \emph{Journal Name} 12(3), 45 -- 678.
\end{thebibliography}
}
\end{frame}

%------------------------------------------------

\begin{frame}
\Huge{\centerline{The End}}
\end{frame}

%----------------------------------------------------------------------------------------

\end{document} 
